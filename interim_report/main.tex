\documentclass[sigconf,nonacm,screen]{acmart}
\geometry{a4paper}

%% PACKAGES %%
\usepackage[utf8]{inputenc}
\usepackage{microtype}
\usepackage[capitalise,nameinlink,noabbrev]{cleveref}
\usepackage{tikz}

\usepackage{lipsum} 

% % % % % % % % % % % % % % % % % % % % %
%          STUDENT INFORMATION 
% % % % % % % % % % % % % % % % % % % % %
\title{The Title of Your Project}

\author{Clever Student} %% ENTER YOUR NAME HERE!!!
\affiliation{
  \country{1234567a} %% ENTER YOUR MATRICULATION NUMBER HERE!!!
}


%% DOCUMENT %%

\begin{document}
% % % % % % % % % % % % % % % % % % % % %
% 			ABSTRACT
% % % % % % % % % % % % % % % % % % % % %

\begin{abstract}
\lipsum[1]
\end{abstract}

%%%%%%%%%%%% DO NOT EDIT THIS PART!!! %%%%%%%%%%%%
%%%%%%%%%%%%%%%%%%%%%%%%%%%%%%%%%%%%%%%%%%%%%%%%%%
\maketitle
\tikz [remember picture, overlay] %
\node [shift={(0.5cm,-0.5cm)}] at (current page.north west) %
[anchor=north west,scale=0.7] %
{\includegraphics{CompSci_logo.pdf}};
%%%%%%%%%%%%%%%%%%%%%%%%%%%%%%%%%%%%%%%%%%%%%%%%%%
%%%%%%%%%%%%%%%%%%%%%%%%%%%%%%%%%%%%%%%%%%%%%%%%%%

% % % % % % % % % % % % % % % % % % % % %
% 			INTRODUCTION
% % % % % % % % % % % % % % % % % % % % %
\section{Introduction}
\label{sec:intro}

This document is the \LaTeX template for submitting the \emph{interim report}
for your MSci project, at the School of Computing Science of the University of
Glasgow. This is an updated version, starting for the academic year 2024/25.
Please make sure to update your personal details, including title, name, and
matriculation number, within the
\verb|%% STUDENT INFORMATION %%| section of the attached \LaTeX
source file \verb|main.tex|.

This template is directly derived from ACM's
\href{https://ctan.org/pkg/acmart?lang=en}{\texttt{acmart}} package; make sure
to consult the corresponding
\href{http://mirrors.ctan.org/macros/latex/contrib/acmart/acmart.pdf}{documentation}
if you face any technical issues, or if you want to explore the full range of
features offered.

Unless you are already an experienced \LaTeX user, perhaps the most
straightforward way to typeset your paper is to work directly on Overleaf; this
is a cloud service (so no need for a local installation) which you can easily
sign up for using your \verb|@glasgow.ac.uk| email. Here is also a very useful
\href{https://www.overleaf.com/learn/latex/Tutorials}{\LaTeX\ tutorial}.

\subsection{Note}
Your report does \emph{not} need not follow exactly the section headings
included in this document - this is only a \emph{suggested} structure. Also,
feel free to add subsections, and further structural elements, as desired.


% % % % % % % % % % % % % % % % % % % % %
% 		   STATEMENT OF PROBLEM
% % % % % % % % % % % % % % % % % % % % %
\section{Problem Statement}
\label{sec:problem}
Clearly state the problem to be addressed in your forthcoming project. Explain
why it would be worthwhile to solve this problem.

\lipsum[1]

% % % % % % % % % % % % % % % % % % % % %
% 	         BACKGROUND
% % % % % % % % % % % % % % % % % % % % %
\section{Background Survey}
\label{sec:background}

Present an overview of related prior work including articles, books, and
existing software products. Critically evaluate the strengths and weaknesses of
the previous work.

Perhaps you want to cite the seminal paper of \citet{Turing1937}, or
prior~\cite{Goedel1931} and concurrent~\cite{Church1936} work.

\lipsum[1]


% % % % % % % % % % % % % % % % % % % % %
% 			PROPOSED APPROACH
% % % % % % % % % % % % % % % % % % % % %
\section{Proposed Approach}
\label{sec:approach}

State how you propose to solve your problem. Show that your proposed approach is
feasible, but identify any risks.

\lipsum[1]


% % % % % % % % % % % % % % % % % % % % %
% 	       PRELIMINARY RESULTS
% % % % % % % % % % % % % % % % % % % % %
\section{Preliminary Results}
\label{sec:results}

Our results are summarized in~\cref{tab:table1}, and a visual representation of
our analysis can be seen in~\cref{fig:alice}.

%% Example of table
\begin{table}
\footnotesize
\begin{tabular}{lll}
\toprule
         & machine A                   & machine B                           \\
\midrule
CPU      & Intel Core i7-9700 CPU      & 2x Intel Xeon E5-2630 v3            \\
CPU Frequency& 3.00GHz                     & 2.40GHz                             \\
RAM      & 16GB DDR4                   & 128GB                               \\
OS       & Ubuntu 20.04 LTS            & Ubuntu 16.04 LTS                    \\
Compiler & GCC 9.3                     & GCC 7.3                             \\
libm     & v2.31                       & v2.23                               \\
libomp   & v4.5                        & v4.5                                \\
\bottomrule
\end{tabular}
\caption{This is the table caption.}
\label{tab:table1}
\end{table}

%% Example of figure
\begin{figure}
    \centering
    \includegraphics[width=0.8\columnwidth]{alice.pdf}
    \caption{This is the figure caption.}
    \label{fig:alice}
\end{figure}

\lipsum[1]


% % % % % % % % % % % % % % % % % % % % %
% 			WORK PLAN
% % % % % % % % % % % % % % % % % % % % %
\section{Work Plan}
\label{sec:plan}

Discuss how you plan to organize your work in until the submission of your final
paper, identifying intermediate deliverables and milestones.

\lipsum[1]


% % % % % % % % % % % % % % % % % % % % %
% 			BIBLIOGRAPHY
% % % % % % % % % % % % % % % % % % % % %
\bibliographystyle{ACM-Reference-Format}
\bibliography{refs}
\end{document}
